%\section{Experiment Setup}

%We will begin by measuring the baseline traffic from the smart speaker in a silent environment. This should allow us to identify the regular traffic patterns of the smart speaker when it is not transmitting voice data. %We will then test several basic pre-recorded commands with the smart speaker to identify the basic traffic signature of the speaker when it is transmitting voice data.
%
%Once we have baseline traffic patterns we will begin more detailed measurements. We will test pre-recorded commands of varying lengths many times, to establish correlations between recording length and traffic speed and/or duration. We will then repeat these tests in the presence of several different forms of background noise, including at least podcasts, music, and discussion among ourselves. %We will not perform any tests in public environments to avoid any possible ethical considerations.
%
%To determine how the speaker decides to terminate its recording, we will use several different pathological commands. We will use incomplete commands, very long commands, and commands spoken very unclearly. By correlating the traffic patterns of these commands to the more normal commands previously measured, we hope to calculate the rough length of time the recording lasts.
%
%We will obtain whatever information we can from the cloud (e.g. through an app or API) to compare such information to our traffic pattern analysis and determine whether they match within whatever margin of error we can achieve. We will also test the behavior of the speaker in noisy environments without any commands being given to determine the rate of false recordings.
%
%\subsection{Devices}
%Our main device for testing will be the Amazon Echo, as it is the most popular smart speaker.
%
%If available, we would also like to test the Apple HomePod, as Apple places a much greater emphasis on security and privacy than its competitors.

\section{Experiment}

We ran some experiments on Alexa and use Wireshark to see if Alexa recorded consumer's voice more than we suppose it does.

\subsection{Monitoring Set Up}

In order to catch the packets that were sent out by Alexa to the Amazon cloud server, we set up a WiFi router on a Raspberry Pi and connected the Echo to it. We run Wireshark~\cite[wireshark] on the Raspberry Pi to capture all the packets to and from the Echo speaker.

Since we cannot directly examine the contents of the packets sent to and from the Echo, as they are all encrypted by TLS, we instead must examine coarser-grained information, like packet count, packet sizes, server ip addresses, and packet timing information. Since most of this kind of coarser-grained information is subject to variation based on network conditions, and the voice data itself is subject to variation based on environmental noise, we opt to repeat all of our experiments many times (10s to 100s) to draw statistical conclusions. 

To ensure the repeatability of such experiments, we synthesize all our voice commands and play them to the Echo using the Raspberry Pi. We also use it record any response that the Echo may give, which allows us to gather all relevant data for our experiments on a single, centralized platform. Each experiment consists of some number of different voice commands that we iterate through throughout the experiment. We leave 30 seconds to a minute between each command to allow the Echo to reset, and after reaching the end of our set of commands, we loop back to the beginning. We perform each experiment overnight in the office, to remove any extra human noise and ensure that the Echo only responds to the commands we provide it. We interleave the commands that we are testing to make sure that any variation throughout the night due to conditions we cannot account for affect each of the commands equally.

To make analysis easier, we use the Raspberry Pi to inject markers into the Wireshark packet capture at critical moments. This allows the analysis code to read only the single Wireshark capture and still obtain all necessary information. In particular, the Raspberry Pi pings four different predefined locations (\textit{e.g.}, 8.8.8.8) at four specific times: when the command starts playing, when the command stops playing, when Alexa's response begins, and when Alexa's response ends.

\subsection{Analysis Techniques}

After an overnight run of an experiment, we programmatically divide the Wireshark capture into experiment frames (individual command trials). We define these frames using the ping packets generated by the Raspberry Pi; the extent of each frame spans from the start of the voice command to the start of Alexa's response. 

\subsection{Baseline Experiment}

In the baseline experiment, we kept Alexa in the office overnight without doing anything and caught up all the package that sent out by Alexa.

In this part, we assume that Alexa will not transfer any voice message to the cloud as there were no voice except some noise generated by nature. Therefore, we could use the package that caught up in this experiment to be the baseline, who were the "necessary" package that Alexa communicate with cloud server. And we assume them to be unsuspicious package, which include some handshake package and broadcast package to let the Alexa confirm which server to communicate with.

\subsection{"Prefix" Experiment}

As Amazon has mentioned in his document \todo{[1]}: "By default, Alexa-enabled devices only stream audio to the cloud if the wake word is detected (or Alexa is activated by pressing a button)." it will not transmit any data before the wake up word. In order to test out this announcement, we let Alexa to start transmitting the voice to the server. 

Therefore, in this part, we did two different kinds of experiments. First, we played a voice contained with a prefix sentence as well as a command and another voice without prefix sentence but just command for Alexa. The prefix sentence is a random sentence that simulates a person talking in his daily life, and then he asked Alexa a question. It is really common in our daily life that people just having some conversation and suddenly find something to deal with, therefore they send a command to Alexa. The whole prefix sentence is: " I think I am having some trouble hearing what people trying to say to me. I am not feeling particularly well now. I am also thinking switching a new job." and it played for 9 seconds. The command is: "Alexa, where is New York City." and it played for 2.5 seconds. As a result, if Alexa did record some part of the prefix sentence, the package size of these two parts would show up a significant difference.

Further, we played a voice with the prefix sentence and ranged the time from 9s to 0s, then followed by a 2.5 seconds command to wake Alexa up and let it transmit the package and answer. Our assumption is that, if Alexa truly did the transmission as it mention, the size of each transmission package should be the same, as there is only a short command at the end of the sentence. However, if the optimization algorithm of Alexa is not efficient enough, it might transmit some parts of the prefix sentence to the cloud and the package size would wave all the night.
 

\subsection{"Postfix" Experiment}
  
Moreover, we want to measure whether Alexa will keep transmitting package even if Alexa has noticed the whole command and started replying. The assumption here is that, as Alexa has started replying, it means Alexa has fully recognized the whole command and it should not transmit the rest of the sentence onto the server. The experiment fits our real life situation that someone sends a command to Alexa and starts another conversation with others immediately. If Alexa transmits the rest of the conversation, it would definitely rises the privacy concern.

Here, we use "Alexa, where is New York City" as a command sentence and adds a 1 second postfix sentence. We gave an approximately 0.5 seconds gap between the command and postfix sentence to simulate the normal silent gap when people finished a sentence. Then we started playing three different kinds of sentences which are: a) command voice b) command voice with 0.5 second postfix sentence and c) command voice with 1 second postfix sentence. And analyze the package sent out by Alexa.

What's more, we want to see whether that 0.5 second silence gap between command and postfix sentence is necessary for Alexa to realize the command is over. Here, we played the command with postfix sentence immediately (i.e. delete the silence gap between postfix sentence and command.). We try to find out whether Alexa would automatically cut off the voice once it recognized the following sentence is meaningless. We played command with 0,1,2,3 seconds postfix sentence, which contains nonsense sentences, to see whether Alexa would cut off the transmission itself.

Further, we try to figure out what is the long enough pause for Alexa to cut off the voice transmission and recognize the whole sentence. We played the command with different silence gap before playing the whole postfix sentence, to figure out when Alexa would reply a right answer to our command. Here we tried the silence gap between command and postfix sentence ranged from 0.1 to 0.8 seconds with 0.1 second step.

\subsection{"Stop" Experiment}

Following up the above experiment, in this part, we would like to detect how Alexa detects the end of a command. We would like to see whether Alexa stopped because of silence or something else.

We tried three experiment here: a) We played a background music while playing the command voice and kept playing for a while to see whether Alexa would stop transmitting package and start replying to the command. This experiment can help us figure out whether Alexa will keep recording once they detect some sounds. b) Playing the command voice with a background conversation simulates the situation in our real life. For example, in our family, when father asks something to Alexa, in the meanwhile, mother is talking something to the child. Father's voice is louder however the conversation between mother and child will keep going after father finish his command. Therefore, we would like to figure out whether Alexa can detect this situation automatically. c) We start talking immediately after the wake up world and keep talking for a while without stopping to simulate a situation that in a party there is some conversation happens around Alexa. And we want to figure out whether Alexa would stop recording automatically or will keep recording all of the sentence.



