\begin{abstract}
Smart speakers, such as Amazon Echo, have gained popularity in recent years. These smart speakers normally have always-on microphones that are used to detect voice commands. While being useful, this behavior of constantly listening has raised a wide range of privacy concerns. These include, but are not limited to, what is being recorded, how the collected data is used and stored, and whether it is being protected well. In this work, we quantitatively answered two frequently asked privacy-related questions about Amazon Echo. First, does it stream any conversation before it is activated? Second, is any audio being sent after Echo detects the end of a command? We designed and performed measurements that allowed us to answer these two questions. We first confirm that usually Echo is not transmitting any audio that happens before it is activated. We further verify that in the case of correctly detecting the end of a command, Echo will not stream any conversation that occurs after the command. Additionally, we are able to quantify that a 0.7 second gap is needed for Echo to correctly detect the end of a command. While there is a rich literature on analyzing Echo's network traffic, ours is the first to use its network traffic to answer the two questions aforementioned in a quantitative manner.
\end{abstract}