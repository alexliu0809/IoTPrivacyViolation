\section{Introduction}
Physical devices connected to the Internet, also known as Internet-of-things (IoT) devices, have gained popularity in recent years. Among numerous types of IoT devices, smart speakers with voice assistants, such as the Amazon Echo and the Google Home, are widely seen in users' homes. These devices detect and respond to voice commands. They normally have always-on microphones, which theoretically start recording after hearing a wake word and then send the voice data to a server via Internet for further processing~\cite{AmazonEc68:online}.

This behavior has sparked many privacy concerns, including but not limited to what is being recorded, how the collected data is used and stored, and whether it is being protected well~\cite{lau2018alexa, fowler_2019, apthorpe2017smart, apthorpe2019keeping, apthorpe2017spying}. Much of previous work has been centered around protecting sensitive information from being leaked to adversaries~\cite{apthorpe2017smart, apthorpe2019keeping, apthorpe2017spying}. Other questions, such as what is being recorded and whether Echo is eavesdropping on users or not, remain unanswered. As an attempt to increase transparency, Amazon has made it possible for users to view, play, and delete the voice data transmitted to its server~\cite{ford2019alexa}. However, there is no guarantee that the audio made available by Amazon is all that collected. It is still possible that Echo streams unwanted audio data that happens before or after the command to its server. To our best knowledge, no previous work has tried to measure whether Echo is transmitting anything other than the desired command.

This is where our work comes in. We designed and implemented experiments that allowed us to infer the behavior of the Echo through network traffic analysis. In our experiment setup, we connected an Echo to a dedicated hotspot and used Wireshark to capture all the packets sent and received by Echo. To activate Echo, an audio command was needed. We prerecorded an audio file that contained a simple command ("Alexa, where is New York City?") that was about 2.5~s long. We also prefixed and suffixed the command with irrelevant conversations. This prerecorded audio gave us both repeatability and flexibility. First, by tuning when to start and stop, different segments of this audio could be played. Second, the same segment could be played over and over again, with the guarantee that the input stays the same. We then carried out a set of experiments with this kind of setup.

\textbf{Our first contribution: we confirm that Echo usually only transmits audio that happened after the wake word.} In this experiment, we used two segments from the prerecorded audio. The first audio segment (denoted as $I_1$) was about 2.5~s long and contained only the Echo command "Alexa, where is New York City?". The other audio segment (denoted as $I_2$) was about 11.5~s long, with a 9~s irrelevant conversation attached right before the Echo command. We played both $I_1$ and $I_2$ alternately overnight for over 150 times, giving us in total more than 300 samples. For each sample, after data cleaning, we calculated the total size of all outgoing traffic. We then computed the mean and standard deviation of the total size across all samples of $I_1$ and $I_2$, as shown in Table~\ref{table1}. We did not observe any significant difference in the total size of outgoing packets between $I_1$ and $I_2$. Our hypothesis was that if Echo was sending any audio that happened before the wake word, we should be able to observe different total sizes of outgoing traffic for $I_1$ and $I_2$, which we did not. Thus, we conclude that nothing before the wake word was streamed to the server.

\begin{table}[!b]
\caption{Mean, standard deviation (SD), and median of total size (in kB) of outbound traffic across all samples of $I_1$ and $I_2$}
\begin{tabular}{llll}
\toprule
Audio Input & Mean & SD & Median \\
\midrule
$I_1$ & 44.94 & 0.68 & 44.85\\
$I_2$ & 44.81 & 0.60 & 44.64\\
\bottomrule
\end{tabular}
\label{table1}
\end{table}
% 0-prefix mean: . std: 0.68 median: 44.85
% 9-prefix mean: . std: 0.60 median: 44.64

That being said, we did observe a few outliers that sent significantly more data then expected. Since we were unable to decrypt the traffic, we were unsure if our conclusion still holds for these corner cases.

\textbf{Our second contribution: we characterized Echo's different behaviors when it succeeded in detecting the end of a command and when it failed to do so.} In this experiment, we composed a set of input audios. We started by taking two segments from the prerecorded audio. One was the Echo command (denoted as $I_3$), and the other was a random conversation (denoted as $I_4$) that was about 7~s long. We then crafted a series of input audio $I_5$ to $I_{12}$ by concatenating $I_3$, $x$~seconds of pause, and $I_4$ together, where x varies from 0.1 to 0.8~s with step size 0.1~s. As a result, $I_5$ consists of $I_3$, plus a 0.1~s pause, plus $I_4$. Similarly $I_{12}$ consists of $I_3$, plus a 0.8~s pause, plus $I_4$. We played $I_5$ through $I_{12}$ alternately, as we did in the experiment mentioned above. By looking at the incoming and outbound traffic, we were able to observe the following behaviors:

\textbf{First, we identified that a 0.7~s pause was needed after the command for Echo to successfully sense the end of a command.} For each $x$~second gap, we recorded whether the Echo successfully identified the end of the command ($I_3$) and provided a response while $I_4$ was still playing, or whether instead it waited until the end of $I_4$ to respond, trying to parse the entire input as a command. We saw that below 0.5~s, the Echo never successfully identified the end of the command, and it wasn't until above 0.7~s that the Echo always detected the end of the command.

\textbf{Second, we verified that Echo does not record any audio past the end of a command when such an end is properly detected.} For $I_{11}$ and $I_{12}$, where the pause was clearly long enough for the Echo to detect the end of the command, we observed similar amounts of outgoing traffic as we did for $I_{1}$ and $I_{2}$, implying that Echo did not send anything other the command.

\textbf{Third, if the Echo fails to detect the end of the command, it will continue to record audio.} For $I_{5}$ and $I_{6}$, where the pause was clearly not long enough for the Echo to detect the end of the command, we observed much larger outgoing traffic than we did for $I_{1}$ and $I_{2}$, implying that the Echo did send other audio data than just the command.

In sum, we performed network traffic measurement on the Echo that allowed us to infer the Echo's behavior and address concerns regarding it. We confirmed that the Echo in most cases did not record any audio that happened before the wake word, and that it would not stream any audio after the end of a command if it was able to detect the end. In the case of failing to detect the end of a command, the Echo did send more audio data than just the command to its server.
