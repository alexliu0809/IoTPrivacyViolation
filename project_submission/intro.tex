\section{Introduction}
Physical devices connected to the internet, also known as Internet-of-things (IoT), have gained popularity in recent years. Among numerous types of IoT devices, smart speakers with voice assistants, such as Amazon Echo and Google Home, are widely seen in users' home. These devices detect and respond to voice commands. They normally have always-on microphones, which theoretically start recording after hearing a wake word and then send the voice data to a server via internet for further processing~\cite{AmazonEc68:online}.

This behavior has sparked many privacy concerns, including but not limited to what is being recorded, how the collected data is used and stored, and whether it is being protected well~\cite{lau2018alexa, fowler_2019, apthorpe2017smart, apthorpe2019keeping, apthorpe2017spying}. Much of previous work has been centered around protecting sensitive information from being leaked to adversaries~\cite{apthorpe2017smart, apthorpe2019keeping, apthorpe2017spying}. As for what data is being collected, Amazon has made it possible for users to view, play and delete their voice data stored on its server~\cite{ford2019alexa}. However, our intuition is that Amazon may reveal to users only part of the voice data being collected by its smart speakers. Up until now, little has been done to verify whether Amazon is being honest with us or not---is Amazon making public exactly the audio data it collects?

In this work, we performed a series of experiments with Amazon Echo that answered the following questions: 1. Is Amazon Echo recording before 

Our preliminary experiments with Amazon Echo have found that the recorded part of the same command is not consistent across different trials. This stimulates us to question and examine the behavior of Amazon Echo and dig into what is exactly being collected by it. We plan to use network traffic measurement for our work, through which previous researchers have been able to infer users' activities~\cite{apthorpe2017spying}, and establish network signatures for Amazon Echo~\cite{ford2019alexa}. If this approach does work, we will be able to answer a whole set of research questions with it.
%Our expectation of Amazon Echo is that if it is given the same audio input, it should stop recording at the same place every time. Our preliminary experiments with Amazon Echo have found that the recorded voice of the same command is not consistent across different trials, suggesting something different. This stimulates us to question and examine the behavior of Amazon Echo and dig into what is exactly being collected by it. We are hoping to get some insights into this question using network traffic measurement, as previous researchers \cite{apthorpe2017spying} have successfully inferred users' activities leveraging the same technique.

%\section{Goals}

%We want to quantify how far beyond the bounds (start and end) of a particular command the smart speaker records and transmits to the cloud. We want to test how the ambient noise levels (e.g. music or conversation) surrounding the command affect the length of this extra recording.

%We also want to know how the smart speaker decides when to stop recording and transmitting to the cloud. % Does it receive a stop command from the cloud, or does it stop on its own, either through detecting the end of the command through some heuristic or a simple time-out? In particular, if the ending condition is missed for whatever reason, how long will the smart speaker record and upload to the cloud before stopping? \textcolor{red}{We would imagine that Alexa stop recording when they first find a word that should not belong to the command or they just stop recording automatically (or after a few timeouts). We can infer the algorithm that Alexa uses and let user know how to protect their privacy.}
%Finally, %\textcolor{red}{after a test, we find out that sometimes Alexa started recording by a word before the "wake up" command (i.e."Alexa") which made the command meaningless. We want to detect why this will happen and} 
%we want to know if recordings are being made and sent to the cloud without being reported to the user by the service provider (e.g. through an app or API).