\section{Introduction}
Physical devices connected to the internet, also known as Internet-of-things (IoT), have gained popularity in recent years. Among numerous types of IoT devices, smart speakers with voice assistants, such as Amazon Echo and Google Home, are widely seen in users' home. These devices detect and respond to voice commands. They normally have always-on microphones, which theoretically start recording after hearing a wake word and then send the voice data to a server via internet for further processing~\cite{AmazonEc68:online}.

This behavior has sparked many privacy concerns, including but not limited to what is being recorded, how the collected data is used and stored, and whether it is being protected well~\cite{lau2018alexa, fowler_2019, apthorpe2017smart, apthorpe2019keeping, apthorpe2017spying}. Much of previous work has been centered around protecting sensitive information from being leaked to adversaries~\cite{apthorpe2017smart, apthorpe2019keeping, apthorpe2017spying}. Other questions such as what is being recorded and whether Echo is eavesdropping on users or not remain unanswered. As an attempt to increase transparency, Amazon has made it possible for users to view, play and delete the voice data transmitted to its server~\cite{ford2019alexa}. However, we are not fully convinced that the voice data made available is the only data transmitted by Echo everytime. More specifically, we want to know if Echo is streaming any conversation that happened before the wake word. \todo{and what?} 

To answer all these questions, we opted for a passive mearsurement approach --- analyzing network traffic sent by Echo. We used wireshark to capture all the packets transmitted by Echo. An audio that could trigger Echo to actively send data was also needed. We prerecorded an audio file that contained an Echo command ("Alexa, where is New York City?") that was about 2 seconds long. In this audio file, we also prefixed and suffixed the Echo command with irrelevant conversations. This prerecorded audio file gave us the ability to perform controlled experiments and verify our assumptions. Different segments of this audio were played to Echo over and over again, and each time we would compute the total package size transmitted. Having done this, for each unique segment of the audio, we now had a set of total packet sizes that Echo transmitted when hearing this segment of audio. Then, statistical analysis was to performed to test our assumptions.

\textbf{Our first contribution: we confirmed that most of the times Echo only started to record after hearing the wake word.} To test whether Echo would record the conversation before the wake word or not, we crafted two input audio segments for Echo. The first audio segment (denoted as $I_1$) was about 2.5 seconds long, and contained only the Echo command "Alexa, where is New York City?". The other audio segment (denoted as $I_2$) was about 11.5 seconds long, with a 9 seconds irrelevant conversation attached before the Echo command. We played both $I_1$ and $I_2$ alternatively overnight in a setting with noise level X\todo{do we need noise?}. For each segment, we played it for over 100 times, giving as over 100 sample timeframes, within each one could easily compute the total size of packets transmitted. We manually inspected all the timeframes, removing obvious outliers. We then performed \todo{I did t-test} on these datasets, getting a p-value of \todo{X}. Given that our null hypothesis is they are not the same, we come to the conclusion that Echo only started to record after hearing the wake word.

\textbf{Our second contribution: we confirmed that the Echo does not record any audio past the end of a command when such an end is properly detected. However if the Echo fails to detect the end of the command, it will continue to record audio.}

\textbf{Our third contribution: we identify the necessary pause detection of the end of a command, and determine that such detection is done locally on the Echo and not on the Amazon servers.}



\todo{Other experiments we did.}

%\section{Goals}

%We want to quantify how far beyond the bounds (start and end) of a particular command the smart speaker records and transmits to the cloud. We want to test how the ambient noise levels (e.g. music or conversation) surrounding the command affect the length of this extra recording.

%We also want to know how the smart speaker decides when to stop recording and transmitting to the cloud. % Does it receive a stop command from the cloud, or does it stop on its own, either through detecting the end of the command through some heuristic or a simple time-out? In particular, if the ending condition is missed for whatever reason, how long will the smart speaker record and upload to the cloud before stopping? \textcolor{red}{We would imagine that Alexa stop recording when they first find a word that should not belong to the command or they just stop recording automatically (or after a few timeouts). We can infer the algorithm that Alexa uses and let user know how to protect their privacy.}
%Finally, %\textcolor{red}{after a test, we find out that sometimes Alexa started recording by a word before the "wake up" command (i.e."Alexa") which made the command meaningless. We want to detect why this will happen and} 
%we want to know if recordings are being made and sent to the cloud without being reported to the user by the service provider (e.g. through an app or API).