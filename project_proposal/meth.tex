\section{Goals}

We want to quantify how far beyond the bounds (start and end) of a particular command the smart speaker records and transmits to the cloud. We want to test how the ambient noise levels (e.g. music or conversation) surrounding the command affect the length of this extra recording.

We also want to know how the smart speaker decides when to stop recording and transmitting to the cloud. % Does it receive a stop command from the cloud, or does it stop on its own, either through detecting the end of the command through some heuristic or a simple time-out? In particular, if the ending condition is missed for whatever reason, how long will the smart speaker record and upload to the cloud before stopping? \textcolor{red}{We would imagine that Alexa stop recording when they first find a word that should not belong to the command or they just stop recording automatically (or after a few timeouts). We can infer the algorithm that Alexa uses and let user know how to protect their privacy.}
Finally, %\textcolor{red}{after a test, we find out that sometimes Alexa started recording by a word before the "wake up" command (i.e."Alexa") which made the command meaningless. We want to detect why this will happen and} 
we want to know if recordings are being made and sent to the cloud without being reported to the user by the service provider (e.g. through an app or API).

\subsection{Devices}
Our main device for testing will be the Amazon Echo, as it is the most popular smart speaker.

If available, we would also like to test the Apple HomePod, as Apple places a much greater emphasis on security and privacy than its competitors.

\section{Methodology}

We will begin by measuring the baseline traffic from the smart speaker in a silent environment. This should allow us to identify the regular traffic patterns of the smart speaker when it is not transmitting voice data. %We will then test several basic pre-recorded commands with the smart speaker to identify the basic traffic signature of the speaker when it is transmitting voice data.

Once we have baseline traffic patterns we will begin more detailed measurements. We will test pre-recorded commands of varying lengths many times, to establish correlations between recording length and traffic speed and/or duration. We will then repeat these tests in the presence of several different forms of background noise, including at least podcasts, music, and discussion among ourselves. %We will not perform any tests in public environments to avoid any possible ethical considerations.

To determine how the speaker decides to terminate its recording, we will use several different pathological commands. We will use incomplete commands, very long commands, and commands spoken very unclearly. By correlating the traffic patterns of these commands to the more normal commands previously measured, we hope to calculate the rough length of time the recording lasts.

We will obtain whatever information we can from the cloud (e.g. through an app or API) to compare such information to our traffic pattern analysis and determine whether they match within whatever margin of error we can achieve. We will also test the behavior of the speaker in noisy environments without any commands being given to determine the rate of false recordings.
