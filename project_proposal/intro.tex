\section{Introduction}
Phsycial devices connected to the interent, also known as Internet-of-things (IoT), have gained its popularity in recent years. Among numerous types of IoT devices, smart speakers with voice assitants, such as Amazon Echo and Google Home, are widely seen in users' home. These devices would detect and respond to voice commands. They normally have always-on mircophones, which would start recording after hearing a wake word (Alexa, Echo, etc.) and then send the collected data to a server through internet for further processing \cite{AmazonEc68:online}. This behaviour has sparked many privacy concerns, including but not limited to what is being recorded, how the collocted data is used and stored, and whether it is being protected well \cite{lau2018alexa, fowler_2019, apthorpe2017smart, apthorpe2019keeping, apthorpe2017spying}. Much of previous work has been centered around protecting sensitive information from being leaked to adversaries\cite{apthorpe2017smart, apthorpe2019keeping, apthorpe2017spying}. However, up until now, little has been done to investigate what is being recorded by smart speakers and whether they stop recording when they are expected. Our intuition is that these smart speakers would record for an extra amount of time and thus leaking some information. The problem is further exacerbated by the fact that Amazon would keep the voice data being sent to them \cite{kelly_statt_2019, osborne_2019}. This work examines the question of what is being recorded by smart speakers through network traffic measurement. Previous researchers \cite{apthorpe2017spying} have successfully inferred user activies using this technique. We are hoping to get some insights into this question by performing similar experiments.